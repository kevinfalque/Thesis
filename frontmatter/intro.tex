% !TeX encoding = UTF-8
% !TeX spellcheck = fr_FR
% !TeX root = mythesis.tex

\chapter*{Introduction}

%%% the two following lines are needed for  starred chaper 
%%% in main matter and want it to appear in  TOC ans bookmarks
\phantomsection
\addcontentsline{toc}{chapter}{Introduction}
%%% the next lne is needed to update the headers dor a starred chapter
\markboth{INTRODUCTION}{INTRODUCTION}

\section*{General context}

The last thirty years have been marked by what has been dubbed the "second quantum 
revolution".
Physicists have aimed at switching from the understanding of the foundations of quantum
mechanics to actually harnessing these laws in order to control quantum systems.
In this context, tremendous efforts have been made to develop platforms of controllable
ensembles of quantum particles: cold gases 
\cite{dalibard_fluides_cdf_2016,dalibard_12_superfluidity,pitaevskij_bose-einstein_2016}, 
exciton polariton condensates \cite{amo_polariton_2011,Amo_fluidlightexp_2009, kasprzak_boseeinstein_2006}, 
superconducting circuits \cite{Wendin_2017}, 
dye-filled cavities \cite{Kurtscheid2019-vw, Ozturk2022-hp, Nyman2021-pd}
or in our case hot atomic vapors 
\cite{glorieuxHotAtomicVapors2023}.

Study of these systems brought experimental confirmation of many seminal effects of the 
many-body quantum physics like Bose-Einstein condensation 
\cite{bec_jila_1995, Ozturk2022-hp, kasprzak_boseeinstein_2006}, 
superfluidity \cite{dalibard_12_superfluidity}, 
superconductivity \cite{strinati_bcsbec_2018} or 
quantum entanglement \cite{josse_entanglement_2004}.
These effects were subsequently harnessed to great success bringing considerable 
technological advancement for medical imagery thanks to superconducting magnets \cite{mri_magnets},
guidance systems with the advance of atomic clocks \cite{optical_clocks} for the GPS and 
communications with fully optical links \cite{fiber_amplifiers}.

In this effervescence, quantum fluids of light
have emerged as a competitive option due to their relative simplicity and excellent 
imaging capabilities comparatively to cold gases for instance, where non destructive measurement of 
the system is difficult.
In fluids of light, the quantum particles are directly photons whose control has been 
perfected by hundreds of years of optics, both classical, and more recently quantum.
This grants near unlimited access to the full field, making fluids of light the only 
platform where the correlation functions of all moments can be computed directly.
However, photons do not interact in a vacuum.
In order to obtain a collective behavior in an ensemble of photons, one needs to engineer
interactions between them.
In a medium, photons acquire an effective mass due to the index of refraction of the 
medium.
If in addition this medium has an intensity dependant nonlinear index, the 
photons will acquire effective interactions.
Under the right conditions, if the interactions are repulsive, the photons will behave
collectively, exhibiting fluid-like behavior.

Such a nonlinear medium can be found in hot atomic vapors that have several key advantages
to other nonlinear media:
\begin{itemize}
    \item Compared to photorefractive crystals \cite{michel_superfluid_2018,michel_phonon_2016}, 
    they are repeatable in the sense that two cells 
    containing the same element will behave in the same manner, while photorefractive
    crystals are by nature flawed by large variations of their properties between samples.
    Also, the non-linearity of atoms can be several orders of magnitude larger and this 
    non-linearity can be tuned over several orders of magnitude through the exponential
    dependance of the atomic density on temperature.

    \item Compared to exciton polaritons, hot atomic vapors do not require vacuum chambers and 
    cryostats in order to bring the samples to their working conditions. 
    Furthermore these samples require very advanced nanofabrication techniques that 
    forbid fast iterations between samples.
    While the coupling between light and excitons in polariton cavities is exceptionnal, 
    it can hardly be tuned.
    Furthermore, exciton polaritons fluids are driven dissipative, meaning that there is 
    a complex exchange between the exciton reservoir, cavity photons and polaritons.
    This makes interpretation of the experiments more challenging.
\end{itemize}

This platform has proven very successful allowing to probe hydrodynamical effects from 
vortex dynamics \cite{azam_2022}, dispersive shock waves \cite{bienaime_quantitative_2021} 
up 
to large scale hydrodynamical phenomena like turbulence \cite{abobaker2022inverse}. 
Statistical properties of quantum fluids were also observed shedding light on the
response to interaction quenches \cite{steinhauer_analogue_2022} as well as prethermalization
\cite{Abuzarli_2022}.

These results and the fruitful collaboration between theory and experiments 
shows great promise for the future of our community. 

\section*{Motivations}

The hydrodynamical properties of fluids of light in hot atomic vapors were first demonstrated in \cite{fontaine_phd} 
where Quentin Fontaine laid the ground work of the experiments carried out in this thesis:
the first methods to measure the Bogoliubov dispersion were developped and the interference of Bogoliubov 
particles was evidenced.
In \cite{abuzarli_phd}, Murad Abuzarli introduced new methods to precisely characterize the nonlinear index of 
refraction, as well as observables in order to probe the out-of-equilibrium nature of paraxial fluids of light
presenting blast wave dynamics as well as studies on the coherence properties of the fluid after a quench.
The motivation of this thesis is to build on these works and mature the experimental set up into 
a fully controlled sandbox.
In order to provide a credible alternative to the leading experimental platforms like cold atoms, we
build a theoretical, numerical and experimental framework in order to be able to predict ab initio, 
simulate and put into practice the Hamiltonian we want to study.

The goal of reaching this full control of the system is also to address important theoretical questions of 
paraxial fluids of light: the system dimensionality (2D or 3D), the meaning of the space-time mapping or the limits 
of the Bogoliubov theory.
We can thus summarize the motivation of this thesis as an attempt to from passive observation to active planning 
and control of a specific physical effect.

\section*{Summary}

This thesis manuscript contains five chapters, structured in increasing scale order: from the lowest quantum level scale 
to the macroscopical large scale.
The common thread across chapters is the system Hamiltonian: 
in each chapter, we focus on the description and control of one term.
This corresponds to the degrees of freedom of the fluid (kinetic, interaction and potential energy) and we ultimately
link these to experimental tuning knobs.
In each chapter, we try to present the theoretical framework, numerical methods and experimental results in 
order to try and paint a complete picture.
The first chapter is devoted to introducing the theoretical concepts and description of paraxial fluids of light.
The second chapter explores the effects arising from quantum fluctuations of vacuum and interaction quenches.
The third chapter describes the atomic medium that generates the all important photon-photon interactions.
The fourth chapter presents a detailed study of superfluidity. 
Finally the fifth chapter opens up new perspectives on energy transfer accross scales presenting results on 
turbulence and quantum phase transitions.

\paragraph*{Chapter 1}

The theoretical framework used to describe paraxial fluids of light starting from the description of light
propagation in nonlinear media.
After establishing the nonlinear Schrödinger equation, the quantum formulation is introduced and the 
Bogoliubov theory is presented.
A detailed comparison with the Gross Pitaevskii equation of cold gases is presented.
The last section presents the numerical analysis tools used to solve the evolution equation of the system.

\paragraph*{Chapter 2}

The impact of the interaction quenches on the statistical properties of the fluid is presented.
The evolution equation is solved analytically and we present an experimental measurement of the 
predicted structure factor.
We present a novel experimental technique to measure the structure factor using a balanced homodyne 
detection. 
We conclude by presenting the expected signatures of effects that highlight scattering processes 
between Bogoliubov excitations, going beyond the theory presented in chap.\ref{chap:quantumfol}.

\paragraph*{Chapter 3}

The structure of the Rubidium atom is presented and the interaction with the electric field is derived.
A three-level model is presented, before extending it to four levels and describing optical pumping.
Transit effects due to the thermal motion of atoms are then explored with Monte Carlo numerical 
simulations.
Finally, we propose two new configurations in order to tune the atomic response using electromagnetically 
induced transparency (EIT).

\paragraph*{Chapter 4}

In this chapter we explore the emergence and breakdown of superfluid flow in fluids of light.
We start by a comparison of the relevant observables to identify the superfluid transition.
We present the scattering experiment that we will use to probe the critical velocity for 
superfluidity and detail the measurement methods used to characterize the defect induced in 
the fluid.
We then derive a more refined model in order to describe the back-reaction of the fluid 
on the defect and use this back-reaction to obtain a time resolved measurement of the drag force.
We use this coupled model to numerically predict the critical velocity for various defects 
and compare this to experimental results.
We conclude by presenting measurements of the Bogoliubov dispersion relation hinting at 
nonlocal behavior in the fluid.

\paragraph*{Chapter 5}

We conclude this thesis by presenting recent results on turbulent behavior in fluids of light.
We start by introducing the dynamical instability phenomenon giving rise to the turbulent 
behavior.
We then present experimental results of vortex clustering and inverse energy cascade that
are trademarks of turbulence.
We then study more closely the vortex dynamics as individual objects by deriving the 
Berezinskii-Kosterlitz-Thouless (BKT) transition, extracting the vortex interaction 
potential and present experimental measurements of this interaction potential.
We finish by an experimental proposal to try and evidence the BKT transition by monitoring the
vortex correlations and clustering.
Finally, we go back to the fluid of light and explore the thermodynamics of the fluid
by using the beyond mean field predictions of chap.\ref{chap:static}.