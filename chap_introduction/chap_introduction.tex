% !TeX encoding = UTF-8
% !TeX spellcheck = fr_FR
% !TeX root = ../mythesis.tex
% !TeX program = pdflatex (build)
%%% TeXmaker : no 'magic comments' but set Root with Options > Set as master file


\chapter{Introduction}
\label{chap:introduction}

Studying quantum effects in gravitational systems remains one of the most challenging endeavors in modern physics. 
The extreme conditions required to probe such phenomena, such as those near black holes, make direct experimental investigations practically impossible. 
Among these quantum effects, Hawking radiation stands out as a cornerstone prediction of quantum field theory in curved spacetime. Proposed by Stephen Hawking in 1974, this radiation arises from the interplay between quantum mechanics and general relativity, predicting the emission of thermal radiation from the event horizon of black holes. Despite its profound implications, Hawking radiation has yet to be observed directly due to the faintness of the signal and the difficulty of accessing black hole horizons.
 The obvious difficulty to test this prediction experimentally is double. First, the blackness of such an object makes it hard to spot with a telescope meaning one have to rely rather on the peculiar behavior of visible object moving in its gravitationnal field. The optical observation of a black hole took almost one century since the first prediction of gravitationnal collapse by Subrahmanyan Chandrasekhar in 1920. It required 
the synchonization of nine telescope across the world to obtain an optical system whose optical aperture is the the diameter of earth. Then the black body temperature of \textcolor{red}{FINIR} of a black hole is inversely proportional to its mass, making Hawking radiation extremely weak for astrophysical black holes.

To circumvent these challenges, William Unruh introduced the concept of analog gravity in 1981, proposing that certain fluid systems could mimic the behavior of spacetime near a black hole. 
This analogy relies on the mathematical equivalence between the propagation of sound waves in a moving fluid and scalar fields in curved spacetime. 
Since then, numerous classical analog gravity experiments have been conducted, demonstrating phenomena such as horizons and superradiance in systems ranging from water waves to optical fibers. While these experiments have provided valuable insights into the classical aspects of analog gravity, they fall short of capturing the quantum nature of effects like Hawking radiation.

In recent years, the focus has shifted toward quantum fluids, which offer the unique advantage of enabling the observation of quantum effects in analog gravity systems. Among these, quantum fluids of light have emerged as particularly promising candidates. 
These systems, formed by the interaction of photons with a nonlinear medium, combine the advantages of quantum fluids—such as coherence and tunability—with the accessibility of optical techniques. Quantum fluids of polaritons, which arise from the strong coupling between photons and excitons in semiconductor microcavities, are especially well-suited for analog gravity experiments. 
Their out-of-equilibrium nature, combined with their ability to form horizons and exhibit quantum fluctuations, makes them ideal platforms for studying phenomena like Hawking radiation.

This thesis explores the theoretical and experimental investigation of analog Hawking radiation in polariton quantum fluids. 
By leveraging the unique properties of these systems, we aim to bridge the gap between classical analog gravity experiments and the quantum realm, providing new insights into the interplay between quantum mechanics and curved spacetime physics.

  
In the \textbf{second chapter}, we introduce polariton fluids, which result from the strong coupling between photons and excitons in semiconductor microcavities. We present the fundamental properties of these systems, including their out-of-equilibrium nature, coherence, and ability to form analog horizons. These characteristics make them ideal candidates for analog gravity experiments. 
We also establish the theoretical framework based on the Gross-Pitaevskii equation, which describes the collective dynamics of polariton fluids.

\textbf{Chapter 2: Theory of Analog Hawking Radiation}  
This chapter develops the theoretical framework for understanding analog Hawking radiation in polariton fluids. By deriving the Bogoliubov spectrum of collective excitations, we highlight the role of positive- and negative-energy modes in particle creation. 
We analyze the mixing of these modes at the transcritical interface, which is the key mechanism behind Hawking radiation emission. Finally, we discuss the theoretical implications of this phenomenon and the conditions required for its observation.

\textbf{Chapter 3: Experimental Realization of a Transcritical Flow}  
In this chapter, we describe the experimental realization of a transcritical flow in a polariton fluid. Through precise optical pump shaping, we successfully created a transcritical region, essential for the emergence of a sonic horizon. We also present an experimental method to detect negative-energy modes in this region, confirming the necessary conditions for observing analog Hawking radiation.

\textbf{Chapter 4: Experimental Observation of Stimulated Hawking Radiation}  
This chapter focuses on the experimental observation of stimulated Hawking radiation in a polariton fluid. By injecting a coherent state into the upstream region and measuring the scattered modes, we provide evidence of positive-negative energy mixing at the interface. While our results do not yet constitute a complete quantitative measurement of amplification, they demonstrate the power of optical techniques for reconstructing the fluctuation field and analyzing scattered modes.

\textbf{Chapter 5: Perspectives and Conclusions}  
In the final chapter, we discuss the perspectives offered by polariton fluids for analog gravity experiments. We emphasize the necessary improvements for a complete quantitative study, including the reconstruction of the scattering matrix and the analysis of correlations between emitted modes. These advancements would deepen our understanding of the quantum aspects of Hawking radiation. Finally, we conclude by highlighting the unique potential of polariton fluids as experimental platforms for exploring fundamental phenomena in quantum field theory in curved spacetime.