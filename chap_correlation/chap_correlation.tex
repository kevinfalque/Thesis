% !TeX encoding = UTF-8
% !TeX spellcheck = fr_FR
% !TeX root = ../mythesis.tex
% !TeX program = pdflatex (build)
%%% TeXmaker : no 'magic comments' but set Root with Options > Set as master file

%useful stuff for what follows

\graphicspath{{./}{./fig/}{./chap_correlation/fig/}}

\chapter{Bogoliubov modes correlations in polariton quantum fluid}

\label{chap:correlation}

The study of collective excitations in quantum fluids is fundamental to understand nonequilibrium dynamics and many-body interactions. Furthermore, the analog Hawking radiation
on which this work is focused, is expected to create non classical correlations between bogoliubov modes, the event horizon acting as a  two mode squeezer. The simplest manifestation
of paired correlated emission can be observed in the density fluctuations second order correlation function \cite{nguyen_acoustic_2015, carusotto_stimulatedfluid_2016,jacquet_quantum_2023,steinhauer_observation_2016}.
This observable exhibits correlation pattern in the density fluctuations of the fluid on both side of the horizon.
Yet, this has been done only numerically in polaritonic system since the experimental equivalent would require to resolve the polariton lifetime $\sim 10 \mathrm{ps}$. However 
the strong photonic component of this system suggest that correlations between collective excitations mode must have an optical signature that could be adressed
with all the tools developpped in quantum optics.
In this work, we report the first experimental measurement of correlations between collective excitation modes—Bogoliubov modes—in a static and homogeneous quantum fluid of microcavity exciton-polaritons.
 By using a balanced detection set-up, we measure intensity correlation between the normal and ghost branches, probing the fluctuation dynamics of polariton fluids and extracting the spectral correlations of Bogoliubov excitations.
  We observe a clear enhancement of the intensity correlations when the polariton fluid operates near the turning point of the bistabilty due to the emergence of correlated phonon-like excitations. These correlations, seeded by quantum and thermal fluctuations, provide insights into the role of the nonlinear and phononic interactions in the collective excitations of a polariton quantum fluid.