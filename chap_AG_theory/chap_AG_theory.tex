

% !TeX encoding = UTF-8
% !TeX spellcheck = fr_FR
% !TeX root = ../mythesis.tex
% !TeX program = pdflatex (build)
%%% TeXmaker : no 'magic comments' but set Root with Options > Set as master file

%useful stuff for what follows

\newcommand{\xbf}{\pmb{x}}
\newcommand{\ci}{\mathrm{i}}
\newcommand{\ee}{\mathrm{e}}
\newcommand{\lr}[1]{\left(#1\right)}
\newcommand{\lrsq}[1]{\left[#1\right]}
\newcommand{\tp}{\mathrm{p}}
\newcommand{\tv}{\mathrm{v}}
\newcommand{\vtv}{\boldsymbol{\mathrm{v}}}
\newcommand{\vna}{\boldsymbol{\nabla}}
\newcommand{\vx}{\mathbf{x}}
\newcommand{\tx}{\mathrm{x}}
\newcommand{\vobf}{\pmb{v_0}}
\newtheorem{theorem}{Theorem}
\newtheorem{lemma}[theorem]{Lemma}

\newcommand{\kbf}{\mathbf{k}}
\newcommand{\vbf}{\pmb{v}}
\newcommand{\rbf}{\mathbf{r}}
\newcommand{\ombog}{\omega_{\mathrm{b}}}

\newcommand{\rmexp}{\mathrm{exp}}
\newcommand{\im}{\mathfrak{Im}}
\newcommand{\re}{\mathfrak{Re}}
\newcommand{\mbogo}{m_{\mathrm{det}}}
\newcommand{\cs}{c_{\mathrm{s}}}
\newcommand{\dpsi}{\delta\psi}
\newcommand{\opsi}{\hat{\psi}}
\newcommand{\odpsi}{\hat{\delta\psi}}
% \newcommand{\norm}[1]{\left\lVert#1\right\rVert}
\newcommand{\dr}{\mathrm{d}\mathbf{r}}
\newcommand{\Lbog}{\mathcal{L}_B}


\graphicspath{{./}{./fig/}{./chap_Ag_theory/fig/}}

\chapter{Hawking radiation in a polariton quantum fluid}
\label{chap:AG_theory}

Now that polaritons have been introduced, we can start to discuss the theoretical framework that will be used to 
describe Hawking radiation in a polariton quantum fluid. In this chapter, we will first establish the original hydrodynamical analogy between the propagation 
of acoustic waves in a moving fluid and scalar fields in curved spacetime. This analogy is at the heart of what motivated the study 
of Hawking radiation in analog systems, leading to a first variety of experiment in classical systems such as water tanks. These great experiments,
\cite{rousseaux_observation_2008,weinfurtner_measurement_2011} demonstrated positive to negative frequency conversion of shallow surface waves at the vicinity of a sonic horizon. However the temperature
of classical fluids is too high to quantize their collective excitation field which prevent the study of quantum effects. More precisely, Hawking radiation is expected 
to create entanglement between the emitted modes which can not be tested with water waves. Quantum fluids appear then as promising candidate since the 
fluctuations around the ground or steady state require a quantum treatment to be understood. In fact we will see that tackling a fluid with trans-critical flow with the usual Bogoliubov theory of condensed matter also predict particle creation from vacuum. Studying the effect from this point of view will reveal a strong robustness 
of the effect beyond the original hydrodynamical approach and widen the range of regimes where Hawking radiation can be observed.


\section{The hydrodynamical analogy}

In the previous chapter, we established that the dynamics of microcavity polaritons can be described by a driven dissipative Gross-Pitaevskii equation.
By writting the wavefunction in term of phase and density the Gross-Pitaevskii equation can be cast into a continuity equation and an Euler equation for a non-barotropic fluid in the pump rotating frame. Starting from the GPE
a system pumped by a laser of the general form $F_{\tp}e^{i\phi(x,t)}$ lead the fluid to a stationnary state whose wavefunction $\psi(x,t)=\sqrt{n_0}e^{i\phi(x,t)}$ phase and modulus respect :
\begin{equation}
    \begin{align}
    \partial_t \phi +& \frac{\mlp v_0^2}{2\hbar}+\frac{\hbar}{2\mlp}\frac{\partial_x^2\sqrt{n_0}}{\sqrt{n_0}}+V_{LP}+gn+g_rn_r+\frac{\re\lrsq{F_{\tp}e^{-i\theta}}}{\mlp}=0\,\\
    \partial_t n_0 +& \partial_x(n_0v)= \gamma n_0 -2\im\lrsq{F_{\tp}e^{-i\theta}}\sqrt{n_0}
    \end{align}
\end{equation}
If we neglect the pump and dissipation terms which is basically the situation of a conservative fluid, we recover the original equation describing 
sound waves in a convergent flow. This is precisely the equation that was used in the initial analogy made by W. Unruh \cite{unruh_experimental_1981}.
However, the out of equilibrium nature of the microcavity polaritons brings different phenomenology. In particular, perturbations propagating in
the fluid are not necessary sound waves as we will se latter. It means that the dispersion of these perturbation is not linear and 
can remarkably exhibit a gap which can be associated to massive excitations. To avoid making approximation and encapsulate as much as possible 
the complexity of the system it is convenient to make the same calculation in the pump rotating frame. It equivalently means that all frequencies
will be taken with respect to the pump. In practice this is done by writting the pump $F_{\tp}(x)= F_{\tp}e^{i\theta_p(x)}$ and look for steady state wavefunction 
of the same form $\psi(x,t)=\sqrt{n_0}e^{i\theta_p(x)}$. Plugging this into the GPE we obtain that the phase and density of the fluid must fullfill the following equations :

\begin{equation}
    \lrsq{-\frac{\hbar}{2\mlp}\nabla^2-i\frac{\hbar}{2}(\vna\cdot\pmb{v_0}) -i\hbar(\pmb{v_0}\cdot\vna)- \delta(v_0)+g_{\rm r}n_{\rm r}+g n_0-\ci\frac{\hbar\gamma}{2}}\sqrt{n_0}+|F_\tp|=0\,,
        \label{eq:StatHomogEOSDens}
\end{equation}
with $\hbar \delta(v_0)=\hbar\omega_\tp-\hbar\omega_0-\mlp v_0^2/2$ (note that this is just $\delta(k_p)$ in a homogeneous configuration where $\pmb{v_0}=\hbar\pmb{k_p}/\mlp$).
To study the low energy collective perturbations of the fluid state we linearize the system around the steady state by writing the wavefunction
$\psi(x,t)=(\sqrt{n_0}+\delta\psi(x,t))e^{i\theta_p(x)}$. Injecting this expression in \autoref{reservoir_eq} and using \autoref{eq:StatHomogEOSDens} we obtain at linear order in $\delta\psi$ :

\begin{equation}
    \ci \hbar \lr{\partial_t+\pmb{v_0}\cdot{\vna}}\delta\psi=\lrsq{-\frac{\hbar}{2m^*}\nabla^2+\rho-\ci\sigma}\phi+g n_0 \delta\psi^*,
        \label{eq:DefPertGPE}
\end{equation}
where $$\rho\coloneqq2g n_0- \delta(v_0)+g_{\rm r}n_{\rm r}\qquad\text{and}\qquad\sigma\coloneqq\hbar/2\vna \cdot\pmb{v_0}.$$
Note that here we have assumed that there is no external potential without loss of generality since it can be included in the definition of $\rho$.

\textcolor{red}{Ici il faut finir le calcul et notamment cette histoire de masse et de Klein gordon de mes couilles}

\section{At the heart of particle creation : Bogoliubov transformation}

\subsection{The ambiguity of vacuum definition}
The quantum vacuum state is defined as the quantum state of a system with the lowest possible energy. Eventhough 
this definition is commonly used and looks quite intuitive, it in fact implies some deep properties. First, a minimum energy state exists. In a finite dimension
problem, this is obvious since the spectrum has also a finite dimension but whenever the dimension of the Hilbert space is infinite, the existence of a minimum energy state is a priori not guaranteed.
Secondly, it seems to depend on the system which suggests that the vacuum is not a universal concept. This is particularly true in the context of quantum field theory in curved spacetime where the vacuum state is observer dependent.
For instance, in the Unruh effect, an accelerating observer detect a thermal radiation while an inertial observer sees 'standard' vacuum. This ambiguity is
as at the root of particle creation phenomena like the Hawking effect and once it is understood, make these effects appear less surprising.


\subsubsection{The harmonic oscillator}

To start with, let us consider the usual harmonic oscillator describing a massive particle in a one dimensionnal harmonic potential and
see how the defintion vacuum arise in this simple case \cite{CCT_tome1}. The Hamiltonian of the particle of mass $m$ is :


\begin{equation}
    H=\frac{P^2}{2m}+\frac{1}{2}m\omega^2X^2
\end{equation}
Where $P$ and $X$ are the momentum and position operators satisfying the canonical commutation relation $[X,P]=i\hbar$.
It is convenient to define the dimensionless operators :

\begin{subequations}
    \begin{align}
        \hat{P}&=\frac{1}{\sqrt{\hbar m\omega}}P,\\
        \hat{X}&=\sqrt{\frac{m\omega}{\hbar}}X, 
    \end{align}
\end{subequations}
which respect the commutation relation $[\hat{X},\hat{P}]=i$. The Hamiltonian is then written as :
\begin{equation}
    H=\hbar\omega\hat{H}
\end{equation}
with :
\begin{equation}
    \label{eq:dimensionless_hamiltonian}
    \hat{H}=\frac{1}{2}\lr{\hat{P}^2+\hat{X}^2}.
\end{equation}
The corresponding eigenproblem is usually tackled by introducing the creation and annihilation operators :

\begin{subequations}
    \begin{align}
        \hat{a}&=\frac{1}{\sqrt{2}}\lr{\hat{X}+i\hat{P}},\\
        \hat{a}^\dagger&=\frac{1}{\sqrt{2}}\lr{\hat{X}-i\hat{P}},
    \end{align}
\end{subequations}
which satisfy the commutation relation $[\hat{a},\hat{a}^\dagger]=1$.
Taking the definition of $\hat{a}$ and $\hat{a}^\dagger$ we look a the quantity $\hat{a}^\dagger\hat{a}$ :

\begin{equation}
    \begin{align}
    \hat{a}^\dagger\hat{a}&=\frac{1}{2}\lr{\hat{X}-i\hat{P}}\lr{\hat{X}+i\hat{P}} \\
                          &=\frac{1}{2}\lr{\hat{X}^2+\hat{P}^2+i\lr{\hat{X}\hat{P}-\hat{P}\hat{X}}} \\
                          &=\frac{1}{2}\lr{\hat{X}^2+\hat{P}^2-1}.
    \end{align}
\end{equation}
Comparing with \autoref{eq:dimensionless_hamiltonian} we see that :

\begin{equation}
    \label{eq:ham_number_operator}
    \hat{H}=\hat{a}^\dagger\hat{a}+\frac{1}{2}.
\end{equation}
We then naturally introduce the operator $\hat{N}=\hat{a}^\dagger\hat{a}$. Eventhough this operator 
is known to be the number operator, we will on purpose treat it without prior knowledge and see what can be learned from the derivation.
From \autoref{eq:ham_number_operator} we see that the an eigenstate $\ket{\phi^i_\nu}$ of $\hat{N}$ with eigenvalue $\nu$ is also
an eigenstate of $\hat{H}$ with eigenvalue $\nu+\frac{1}{2}$. Solving the problem then boils down to find the spectrum of $\hat{N}$.

\textbf{Spectrum determination.} The operator $\hat{N}$ is hermitian and its eigenstates form a complete basis of the Hilbert space. Let us 
show two usefull lemmas resulting directly from the definition of $\hat{N}$ :

\begin{lemma}
    The eigenvalues of $\hat{N}$ are positive or zero.
\end{lemma}

\begin{proof}
    Let us consider any eigenstate $\ket{\phi_\nu}$ of $\hat{N}$. The square of norm of the state $\hat{a}\ket{\phi_\nu}$ is 
    \begin{equation}
            \norm{\hat{a}\ket{\phi_\nu}}^2=\bra{\phi_\nu}\hat{a}^\dagger\hat{a}\ket{\phi^i_\nu} \geq 0   
\end{equation}
    Using the definition of $\hat{N}$ we have :
    \begin{equation}
        \label{eq:positive_eigenvalue}
        \bra{\phi_\nu}\hat{a}^\dagger\hat{a}\ket{\phi_\nu}=\nu\bra{\phi_\nu}\ket{\phi_\nu} \geq 0.
    \end{equation}
Since $\bra{\phi_\nu}\ket{\phi_\nu}>0$ we have $\nu\geq 0$.
\end{proof}

\begin{lemma}
    Let $\ket{\phi_\nu}$ be an eigenstate of $\hat{N}$ with eigenvalue $\nu$.
    \begin{itemize}
        \item \ $\nu=0$ if and only if $\hat{a}\ket{\phi_\nu}=0$.
        \item If $\nu>0$ then $\hat{a}\ket{\phi_\nu}$ is an eigenstate of $\hat{N}$ with eigenvalue $\nu-1$.
    \end{itemize}
\end{lemma}

\begin{proof}
    If $\nu=0$, \autoref{eq:positive_eigenvalue} implies that $\bra{\phi_\nu}\ket{\phi_\nu}=0$ which imposes that $\ket{\phi_\nu}=0$.
    Reciprocally, if $\hat{a}\ket{\phi_\nu}=0$ a multiplication by $\hat{a}^\dagger$ gives $\hat{N}\ket{\phi^i_\nu}=0$. Since $\ket{\phi_\nu}$ is not zero we conclude
    that $\ket{\phi_\nu}$ is an eigenvector of $\hat{N}$ with eigenvalue $\nu =0$.

    COnsider now that $\nu>0$, we now know that $\hat{a}\ket{\phi_\nu}$ is not zero. By using the identity $\bigl[\hat{N},\hat{a}\bigr] = -\hat{a}$ we 
    obtain :
    \begin{equation}
        \hat{N}\hat{a}\ket{\phi_\nu}=\hat{a}\hat{N}\ket{\phi_\nu}-\hat{a}\ket{\phi_\nu}=(\nu-1)\hat{a}\ket{\phi_\nu},
    \end{equation}
    which shows that $\hat{a}\ket{\phi_\nu}$ is an eigenstate of $\hat{N}$ with eigenvalue $\nu-1$.
\end{proof}

\noindent Let us summarize what we know so far :
\begin{itemize}
    \item The operator $\hat{N}$ has a non negative spectrum. 
    \item If $\nu$ is an eigenvalue of $\hat{N}$ then $\nu-1$ is also an eigenvalue of $\hat{N}$.
\end{itemize}

This two properties are in fact sufficient to show that the spectrum of $\hat{N}$ are the positive or zero integers. \\
\bigskip 

Assume, by contradiction, that there is a positive non integer eigenvalue $\nu$ associated with the eigenvector $\ket{\phi_\nu}$ . Then, by the second lemma, $\nu-1$ is also a non integer eigenvalue of $\hat{N}$ generated by 
the action of $\hat{a}$ on $\ket{\phi_\nu}$. By iterating this process we obtain a sequence of eigenvalues $\nu,\ \nu-1,\ \nu-2,\dots$ which is impossible since the spectrum is bounded from below. More precisely, it exists
a positive integer $p$ such that $\nu-p<0$ which contradicts the fact that the spectrum is non negative. We conclude that the spectrum of $\hat{N}$ is the positive or zero integers.




\section{Elementary excitations of the fluid : Bogoliubov theory}
Unlike in classical systems, where Hawking radiation can only be triggered by external disturbances, its analog in quantum fluids may instead arise from vacuum fluctuations. In this context, perturbations of the quantum fluid are emitted at the horizon and propagate in opposite directions, mirroring the behavior expected near a gravitational event horizon. 
However, it is important to recognize that at this stage, the analogy with black hole physics—initially derived from hydrodynamical equations—should not be taken as a strict equivalence but rather as an inspiration for exploring particle creation in a quantum fluid.

Indeed, treating the collective excitations of a transonic Bose-Einstein condensate within the framework of Bogoliubov theory already predicts the spontaneous emission of modes at the horizon from vacuum fluctuations. Fundamentally, both Stephen Hawking's original calculation and its quantum fluid analog rely on a Bogoliubov transformation between two sets of operators \cite{hawking_black_1972}, which ultimately leads to the mixing of creation and annihilation operators.

This section will establish the structure of the collective excitation spectrum and demonstrate how specific fluid configurations can give rise to mode emission from vacuum. Furthermore, we will show that the Hawking effect remains robust even when deviating from the original hydrodynamical analogy, and it does not necessarily require the excitations to be sound waves. 
In fact, the intrinsically out-of-equilibrium nature of the polariton system reveals a phenomenology far richer than that of conservative systems, extending the possibility of observing Hawking radiation to massive excitations.



\subsection{Linearization of the Gross-Pitaevskii equation}

Let us consider a monochromatic pump described by a plane wave $F_{\tp}(\rbf,t)=F_{\tp}e^{i(\kbf_p \rbf -\omega_pt)}$. As explained in the previous chapter, this will drive the system
to a steady state of the form $\psi_0(\rbf,t)=\psi_0e^{i(\kbf_p \rbf -\omega_pt)}$ and respecting :

\begin{equation}
    \left[\omp -\omlp - \dfrac{\hbar \kp^2}{2 \mlp} - g \abs{\psilp^0}^2 + i \dfrac{\gamlp}{2} \right] \psilp^0= \eta_{LP} F_p^0.
    \label{eq:steady_state}
\end{equation}

We now look for the elementary excitations following the Bogoliubov prescription which consists in linearizing the GPE around $\psi_0$. We write the wavefunction as :

\begin{equation}
    \psi(\rbf,t)=\biggl[\psi_0(\rbf)+\dpsi(\rbf,t)\biggr]e^{i(\kbf_p \rbf -\omega_pt)}.
    \label{eq:linearization_ansatz}
\end{equation}

The second term in this writting can be interpreted as the polaritons that are not in the steady state and therefore have a global phase different from the pump. 
This imply that they can have different energy and momentum that the fluid and undergo scatterings or spontaneous effects since their dynamics is not fixed by the pump. It is 
worth noticing that this term do not originate from the out of equilibrium nature of the system and would be not zero even at zero temperature due to interactions.
In the case of Bose Einstein condensates, this linearization process actually introduces a state that is not physical in the sense that it suggest the breaking of the $U(1)$ symmetry of the system or equivalently 
 that the condensates picked a phase \cite{castin_bose-einstein_2001}. The true physical state is recovered through a statistical mixtures of all these "symetry broken states" which then rather appear 
as intermediate state convenient for calculation. In the case of a polariton fluid pumped quasi resonantly the situation is different since the mean field phase 
is fixed by the pump. 

\bigskip

Injecting \autoref{eq:linearization_ansatz} in the GPE, using \autoref{eq:steady_state} and keeping only the linear terms in $\dpsi$ we obtain :

\begin{equation}
    \begin{split}
        i  \hbar \partial_t \dpsi =& \biggl(\hbar\omlp^0 - \hbar \omp-\dfrac{\hbar^2}{2\mlp}\left[\nabla^2 +2i\kbf_p\nabla-\kbf_p^2\right]+\hbar g_rn_r + 2\hbar g \abs{\psi_0}^2+ \dfrac{i\hbar\gamma}{2} \biggr) \dpsi \\
                                    &+ \hbar g \psi_0^2 \dpsi^*. 
    \end{split}
    \label{eq:bogo_v1}
\end{equation}

Yet, this equation is not stricly speaking linear in $\dpsi$ since there is a term involving its complex conjugated $\dpsi^*$. A way to solve this problem
is to decompose $\dpsi$ in its real and imaginary part and to obtain two independant equations. An equivalent procedure, more common in the litterature, is to find an equation on 
the complex conjugated and consider $\dpsi^*$ as an independant variable. This is the approach we will follow here. The equation on $\dpsi^*$ is obtained by simple complex conjugation of \autoref{eq:bogo_v1} :

\begin{equation}
    \begin{split}
    -i  \hbar \partial_t \dpsi^* = &\biggl(\hbar\omlp^0 - \hbar \omp-\dfrac{\hbar^2}{2\mlp}\left[\nabla^2 -2i\kbf_p\nabla-\kbf_p^2\right]+\hbar g_rn_r + 2\hbar g \abs{\psi_0}^2- \dfrac{i\hbar\gamma}{2}\biggr) \dpsi^* \\
      &+ \hbar g \psi_0^{*2} \dpsi.
    \end{split}
    \label{eq:bogo_v2}
\end{equation}
For the sake of clarity we define the operator :
\begin{equation}
    \Ham_{bog}= \hbar\omlp^0 - \hbar \omp-\dfrac{\hbar^2}{2\mlp}\left[\nabla^2 +2i\kbf_p\nabla-\kbf_p^2\right]+\hbar g_rn_r + 2\hbar g \abs{\psi_0}^2.
    \label{eq:hambog}
\end{equation}

We can now write the fully linearized problem under its matrix form :

\begin{equation}
    \begin{matrix}i\hbar \partial_t
        \begin{pmatrix}
             \dpsi \\
             \dpsi^*
        \end{pmatrix}
        = (\Lbog  +\dfrac{i\hbar\gamma}{2}\mathrm{I_2})
        \begin{pmatrix}
             \dpsi \\
             \dpsi^*
        \end{pmatrix}
    \end{matrix}
    \label{eq:bogo_matrix}
\end{equation}
where $\Lbog$ is the Bogoliubov matrix defined as :
\begin{equation}
    \begin{matrix}
    \Lbog =
    \begin{pmatrix}
        \Ham_{bog} &  \hbar g \psi_0^2\\
        -\hbar g \psi_0^{*2} & -\Ham_{bog}^*
    \end{pmatrix}
\end{matrix}
\end{equation} 
Since $\Lbog$ is time independant the usual procedure to solve \autoref{eq:bogo_matrix} is to diagonalize the bogoliubov matrix and write any generic solution 
in the basis of eigenmodes defined by :

\begin{equation}
    \Lbog \begin{pmatrix}
        u_{i} \\
        v_{i}
    \end{pmatrix} = (\hbar\omega_i+\dfrac{i\hbar}{2}) \begin{pmatrix}
        u_{i} \\
        v_{i}
    \end{pmatrix}.
\end{equation}
As the entire the spectrum is fully determined by the eigenvalues of $\Lbog$ we will drop the losses term in the following for the sake of simplicity and 
take them into account at the end of the calculation by multiplying the solutions by $\exp(-\gamma t/2)$ :

\bigskip

\textbf{Bogoliubov matrix symmetries.} As in general for quadratic hamiltonian of bosonic systems \cite{castin_bose-einstein_2001} the Bogoliubov matrix is not hermitian. Therefore its eigenvalues are not 
necessarily real and $\Lbog$ is not even ensured to be diagonalizable. However, it can easily be shown that the following symmetry is respected :

\begin{equation}
    \Lbog^\dagger = \eta^{-1} \Lbog \eta \ \ \ \mathrm{with} \ \ \ \eta = \eta^{-1} =\begin{pmatrix}
        1 & 0 \\
        0 & -1
    \end{pmatrix}.
    \label{eq:symetry_bog}
\end{equation}

It means that the Bogoliubov operator is "hermitian" for the inner product :

\begin{equation}
    \langle \vec{X_1},\vec{X_2} \rangle = (\vec{X_1}^*)^{\mathrm{T}} \eta \vec{X_2},
    \label{eq:inner_product}
\end{equation}
It means that for any vectors $\vec{X_1}$ and $\vec{X_2}$ we have $\langle \vec{X_1},\Lbog\vec{X_2} \rangle = \langle \Lbog\vec{X_1},\vec{X_2} \rangle$. 
Consequently the inner product between two modes $\ket{\phi}= (u_{\phi}, v_{\phi})^\mathrm{T}$ and $\ket{\psi}=(u_{\psi}, v_{\psi})^\mathrm{T}$ :

\begin{equation}
    \langle \psi| \phi \rangle_B \coloneqq \int \dr[\psi^\dagger(\rbf)\eta\psi(\rbf)] = \int \dr[u_{\psi}^*(\rbf)u_{\phi}(\rbf)-v_{\psi}^*(\rbf)v_{\phi}(\rbf)],
    \label{eq:inner_product}
\end{equation}
is conserved in time. This product induce a modified norm for the modes : 

\begin{equation}
    \norm{\psi}_{B}\coloneqq \langle \psi| \psi \rangle_B.
    \label{eq:modified_norm}
\end{equation}
It's worth noticing that since the inner product defined in \autoref{eq:inner_product} is neither positive nor definite, the induced norm of a given mode $\ket{\psi}$
can be positive, negative or zero. Let us now see what the symmetry \autoref{eq:symetry_bog} implies on the spectrum of $\Lbog$. Consider an eigenvector $\ket{\psi}= (u_k,v_k)^\mathrm{T}$ of $\Lbog$ with eigenvalue $\hbar\omega_k$ : 

\begin{equation}
    \Lbog \begin{pmatrix}
        u_k \\
        v_k
    \end{pmatrix} = \hbar\omega_k \begin{pmatrix}
        u_k \\
        v_k
    \end{pmatrix}.
\end{equation}
Direct substitution of \autoref{eq:symetry_bog} in the eigenvalue equation gives directly :

\begin{equation}
    \Lbog^\dagger \begin{pmatrix}
        u_k \\
        -v_k
    \end{pmatrix} = \hbar\omega_k \begin{pmatrix}
        u_k \\
        -v_k
    \end{pmatrix}.
\end{equation}
From this, we obtain that $\hbar\omega_k^*$ is also an eigenvalue of $\Lbog$ with since we have the relation :
\begin{equation}
    \mathrm{det}(\Lbog-\hbar\omega_k^*\mathrm{I_d}) = [\mathrm{det}(\Lbog^\dagger-\hbar\omega_k\mathrm{I_d})]^* = 0.
\end{equation}
The Bogoliubov matrix as well as its adjoint operator also respect the symmetry :
\begin{equation}
    \Lbog^* = -\sigma^{-1} \Lbog \sigma \ \ \ \mathrm{with} \ \ \ \sigma = \sigma^{-1} =\begin{pmatrix}
        0 & 1 \\
        1 & 0
    \end{pmatrix},
    \label{eq:symetry_bog_2}
\end{equation}
The latter implies that $-\hbar\omega_k^*$ is also an eigenvalue of $\Lbog$ :

\begin{equation}
    \Lbog \begin{pmatrix}
        v_k^* \\
        u_k^*
    \end{pmatrix} = -\hbar\omega_k^* \begin{pmatrix}
        v_k^* \\
        u_k^*
    \end{pmatrix}.
\end{equation}
which can be again obtained by direct substitution. Putting together all this properties we find that $\hbar \omega_k, -\hbar \omega_k, \hbar\omega_k^*$ and $-\hbar\omega_k^*$ are simultaneously eigenvalues of $\Lbog$.
A special attention must be given to the fact that if $\ket{\psi}=(u_k,v_k)^\mathrm{T}$ is an eigenvector of $\Lbog$ with eigenvalue $\hbar\omega_k$ the eigenvector $\ket{\phi}=(v_k^*,u_k^*)^\mathrm{T}$ linked to $-\hbar\omega_k^*$ has opposite norm.
\begin{equation}
    \langle \psi| \psi \rangle_B = \bra{u_k}\ket{u_k} - \bra{v_k}\ket{v_k} = -\langle \phi| \phi \rangle_B.
\end{equation}
\bigskip


\textbf{Orthogonality condition.} An orthogonality condition can be derived by calculating the quantity $\bra{\psi_i}\eta\Lbog\ket{\psi_j}-\bra{\psi_j}\eta\Lbog\ket{\psi_i}^*$. On one hand we find this term to be zero because 
the symmetry of \autoref{eq:symetry_bog} equivalently means that $\eta\Lbog$ is hermitian. On the other hand we find :

\begin{equation}
    \begin{align}
    \bra{\psi_i}\eta\Lbog\ket{\psi_j}-\bra{\psi_i}\eta\Lbog\ket{\psi_j}^*&=\hbar\omega_i\bra{\psi_i}\eta\ket{\psi_j}-\hbar\omega_j^*\bra{\psi_i}\eta\ket{\psi_j}, \\
    &=(\hbar\omega_i-\hbar\omega_j^*)\bra{\psi_i}\eta\ket{\psi_j},
    \end{align}
\end{equation}
which gives the orthogonality condition :
\begin{equation}
    (\hbar\omega_i-\hbar\omega_j^*)\bra{\psi_i}\eta\ket{\psi_j}=0
\end{equation}
showing that the modified scalar product $\bra{\cdot}\ket{\cdot}_B$ between two eigenvector with different eigenvalues is zero and that 
eigenmodes with complex eigenvalues have a vanishing norm. The modes with non zero norm are then associated to non zero real eigenvalues.


At the end, if we normalize the eigenvectors to unity in the usual sense, the spectrum of $\Lbog$ can be split in three parts :

\begin{itemize}
    \item The $S_+$ family of modes $\ket{\dpsi^+_k}=(u_k,v_k)^\mathrm{T}$ with positive norm : $\bra{u_k}\ket{u_k} - \bra{v_k}\ket{v_k}=+1$ and real eigenvalues $\hbar\omega_k>0$
    \item The $S_-$ family of modes $\ket{\dpsi^-_k}=(v^*_k,u^*_k)^\mathrm{T}$ with negative norm : $\bra{v^*_k}\ket{v^*_k} - \bra{u^*_k}\ket{u^*_k}=-1$ and real eigenvalues $-\hbar\omega_k>0$
    \item The $S_0$ family of modes with zero norm : $\bra{u_k}\ket{u_k} - \bra{v_k}\ket{v_k}=0$ and zero or complex eigenvalues.
\end{itemize}
Note that the eigenvectors of the $S_-$ subspace are expressed in terms of the $u_k$ and $v_k$ component of the $S_+$ vectors to show reflect the dual structure of the solution space. The modes with zero norm are usually related to anormal modes. One of them is the vector whose component are $\psi_0$ and -$\psi_0$ which satisfy the Bogoliubov equations with a zero eigenvalue because 
$\psi_0$ obeys the Gross Pitaevskii equation. It corresponds to global change of the phase condensate and is very similar to the appereance of the Goldstone mode twhen he $U(1)$ phase symmetry of a system is broken. Other modes of $S_0$ are related to dynamical instabilities 
due to non zero imaginary part of their eigenvalues.

\bigskip

\textbf{Dynamical stability.} The free evolution of a mode is given by $\mathrm{exp}(-i\omega_k t)$ so it remains 
bounded in time provided that the imaginary part of $\hbar\omega_k$ is negative or zero. This is known as the dynamical stability condition.
In fact, this condition must be refined in the polaritonic case since as one can see from \autoref{eq:bogo_matrix}, the eigenvalues already 
contain a non zero imaginary part due to the inherent losses in the system. Since eigenvalues come in pairs of complex conjugated the stability condition 
apply both on $\im{(\hbar\omega_k)}$ and $\im{(\hbar\omega_k^*)}=-\im{(\hbar\omega_k)}$ which gives :

\begin{equation}
    \abs{\im{(\hbar\omega_k)}} \leq \dfrac{\hbar\gamma}{2}   \ \ \ \ \mathrm{for \  all  \  k.}
\end{equation}

In practice, the losses can dump dynamical instabilities that
would destroy a conservative system while still exhibiting precursors of the instabilities which make polariton system very usefull for the study of such phenomena 
\cite{claude_high-resolution_2022}. This being said, the present experiments always take place in regime where these modes are not present and the system is stable. 

\bigskip

At the end, if we restrict our description to modes with non zero norm, a general solution of the Bogoliubov equation can be written as :


\begin{equation}
    \ket{\psi}=
    \begin{pmatrix}
    \dpsi \\
    \dpsi^*
    \end{pmatrix} = \sum_{\kbf \in S_+} A_\kbf
    \begin{pmatrix}
    u_k \\
    v_k
    \end{pmatrix}e^{-i\omega_kt}
    + A^*_\kbf   
     \begin{pmatrix}
        v^*_k \\
        u^*_k
        \end{pmatrix}e^{i\omega_kt}
\end{equation}
where 
\begin{equation}
    A_\kbf = \int \dr u_k^*(\rbf)\dpsi(\rbf)-v_k^*(\rbf)\dpsi(\rbf). 
\end{equation}

The redundancy of the first line being the complex conjugated of the second is the price we pay now for our initial assumption that considered $\dpsi$ and $\dpsi^*$ as independant variables. The advantage of the
procedure followed here is the natural emergence of the inner product of \autoref{eq:inner_product} from the symmetries of $\Lbog$. It allowed for the definition of a norm and the establishment of orthogonality among the eigenvectors, 
thereby revealing the structure of the solution space as the direct sum of two dual subspaces. This decomposition results in norms with opposite signs and paired complex-conjugated eigenvalues. Another possible approach would have been
to directly look for solutions as linear combination of plane wave with opposite frequencies as it is done in \cite{pethick_bose-einstein_2008}. This method however does not provide a natural way to define a norm and as we shall
see later, it's the sign of the norm and not the sign of the eigenvalues that is relevant for mode classification. The last advantage of the approach followed here is that the derivation
is closer to the full quantum derivation and will ease the latter quantization of the classical fields obtained here. Taking into account the dissipative part of \autoref{eq:bogo_matrix} and dropping the redundancy a generic
expression for the fluctuations of the order parameter is :

\begin{equation}
    \dpsi(\rbf,t) = \sum_{\kbf \in S_+} \left[A_\kbf u_k(\rbf)e^{-i\omega_kt}+A_\kbf^*v_k^*(\rbf)e^{i\omega_kt}\right]e^{-\gamma t/2}.
    \label{eq:fluctuation_order_parameter}
\end{equation}
\bigskip

\textbf{Particle-hole symmetry.} Under this form one might be tempted to interpret the fluctuations order parameter has a superposition of plane waves with opposite directions and frequencies.
This would be a mistake. Indeed, the coefficient $u_k$ and $v_k^*$ are coupled through the Bogoliubov matrix and respect the particle hole symmetry \autoref{eq:symetry_bog_2}. The first term
can be interpreted as the addition of a particle with energy $\hbar\omega_k$ and momentum $\hbar\kbf$ while the second term can be interpreted as the addition of a hole with energy $-\hbar\omega_k^*$ and momentum $-\hbar\kbf$. The coupling
between the two terms reflect that a fluctuation of the system is particle-hole hybrid excitation, while the norm $\bra{\cdot}\ket{\cdot}_B$ of the mode reflects the relative weight of the particle and hole components.

\begin{itemize}
    \item If $\int \dr[\abs{u_k}^2-\abs{v_k}^2]>0$ the mode is mostly made of $u_k$ and can be interpreted as a particle excitation.
    \item If $\int \dr[\abs{u_k}^2-\abs{v_k}^2]<0$ the mode is mostly made of $v_k^*$ and can be interpreted as a hole excitation.
\end{itemize}

In conservative systems this symmetry is often related to the fluctuation of the number of particles in the condensed state. When the system is out of equilibrium 
this interpretation seems less straightforward since the number of particles in the mean field is not strictly speaking fixed in time. However, these 
modes still account for small deviation from the pump mode $(\kbf_p,\omega_p)$ both in frequency and wavevector. To clarify that, imagine that an experimentalist had the possibility to track simultaneously the incoming laser and 
 its copy that went through the microcavity. Since the linewidth of the laser is much smaller that the microcavity resonance the latter doesn't act as a filter. Hence, the experimentalist would first notice that the two beams share the random events from which the shot noise of the laser originates. 
Then, he would notice that the polariton field exhibit additional fluctuations, namely the random emission of photons with a frequency and wavevector that are not the same as the pump.
These photons originate from the bogoliubov modes of the fluid due to interactions. The photonic nature of this system then appear again as a great asset because the fluctuations of the system
are translated in the noise spectrum of the outcoming laser. 


\subsection{Homogeneous fluid}

So far, the description of the collective excitations spectrum remained quite general and we didn't use the homogeneity of system.
In this case $\psi_0$ is invariant under translation meaning the eigenvectors of $\Lbog$ are just plane waves. The $S_+$ vectors then may be written as :

\begin{subequations}
    \begin{align}
        u_k(\rbf)&=U_k\cdot\mathrm{exp}(i\kbf\rbf) \\
        v_k(\rbf)&=V_k \cdot\mathrm{exp}(i\kbf\rbf).
    \end{align}
\end{subequations}
Before going further in the calculation, let us comment the ansatz of the total field including the fluctuations in terms of reference frame. To obtain the bogoliubov matrix 
we set our description in the rotating frame of the pump by factorizing the total field : $\psi(r,t)=(\psi_0+\dpsi)e^{i(\kbf_p\rbf-\omega_pt)}$. Consequently, all the dynamics on $\dpsi$ we might extract from the equation derived earlier
happen in the moving frame where the fluid is at rest. However, in practice all the accessible observables are measured in the lab frame and are consequently dressed by the motion of the fluid. To directly predict the excitation spectrum as we expect to measure it in the lab
we explictly write the wavefunction as a function of the wavevevector measured in the lab frame $\kbf$ as : $\psi(r,t)=\psi_0e^{i(\kbf_p\rbf-\omega_pt)}+ u_k e^{i\kbf\rbf}$ and look for their dynamics in the pump rotating frame by writting again $\psi(r,t)=(\psi_0+u_ke^{i(\kbf-\kbf_p)\rbf})e^{i(\kbf_p\rbf-\omega_pt)}$. Under this form
the wavevector $\kbf'=\kbf-\kbf_p$ appear as the wavevector of the fluctuations in the moving frame. 
To obtain the corresponding Bogoliubov matrix we first express the action of $\Ham_{bog}$ on the fluctuations plane wave : $u_ke^{i((\kbf-\kbf_p)\rbf)}$ :

\begin{equation}
    \begin{align}
    \Ham_{bog}u_k(\rbf) &= \biggl[\hbar\omlp^0 - \hbar \omp-\dfrac{\hbar^2}{2\mlp}\left[-(\kbf-\kbf_p)^2 -2\kbf_p(\kbf-\kbf_p)-\kbf_p^2\right]+\hbar g_rn_r + 2\hbar g n_0 \biggr]u_ke^{i(\kbf-\kbf_p)\rbf} \\
            &= \biggl[-\hbar\delta(k_p)+\dfrac{\hbar^2(\kbf-\kbf_p)^2}{2\mlp}+\hbar g_rn_r + 2\hbar g n_0\biggr]u_ke^{i(\kbf-\kbf_p)\rbf} + \dfrac{\hbar^2\kbf_p(\kbf-\kbf_p)}{\mlp}u_ke^{i(\kbf-\kbf_p)\rbf}.
    \end{align}
    \label{eq:ham_bog_plane_wave}
\end{equation}

where $\delta(k_p)=\omega_p - \omlp^0 - \dfrac{\hbar k_p^2}{2\mlp}$ is the detuning between the pump and the LP polariton dispersion at $\kbf_p$ provided we stay in the low wavevector
parabolic approximation. The second term proportionnal to $(\kbf-\kbf_p)^2$ shows that due to the fluid motion, the Bogoliubov spectrum observed in the lab frame is as expected
centered on the fluid wavevector. The last term describe the temporal frequency shift due to the fluid motion  $\Delta\omega =\hbar^2\kbf_p(\kbf-\kbf_p)/\mlp=\hbar\vbf_0(\kbf-\kbf_p)$ where $\vbf_0=\hbar\kbf_p/\mlp$ is the fluid velocity. This term is precisely
the term that emerge naturally in the Doppler effect demonstration.
Doing the same procedure for the complex conjugated the Bogoliubov matrix can be written as :



\begin{equation}
    \begin{aligned}
    \Lbog = \ &
    \begin{pmatrix}
        -\hbar\delta(k_p)+\dfrac{\hbar^2(\kbf-\kbf_p)^2}{2\mlp}+\hbar g_rn_r + 2\hbar g n_0 &  \hbar g n_0e^{2ik_p} \\
        -\hbar g n_0^{-2ik_p} & \hbar\delta(k_p)-\dfrac{\hbar^2(\kbf-\kbf_p)^2}{2\mlp}-\hbar g_rn_r - 2\hbar g n_0
    \end{pmatrix}\\
    &+\begin{pmatrix}
        \hbar \vbf_0(\kbf-\kbf_p) & 0 \\
        0 & \hbar \vbf_0(\kbf-\kbf_p)
    \end{pmatrix}\\
    = \ &\Lbog^{'} + \hbar \vbf_0(\kbf-\kbf_p)\mathrm{I_2}
    \end{aligned}
\end{equation}
where the doppler effect appears naturally in the second diagonal matrix and shifts the spectrum of $\Lbog^{'}$ in the fluid rest frame. 
Looking at the roots of the $\mathrm{det}(\Lbog^{'}-X\mathrm{I_2})=0$ we find the eigenvalues in the fluid frame through :

\begin{equation}
    (\hbar\omega'_k)^2 = \left(\dfrac{\hbar^2(k-k_p)^2}{2\mlp}-\hbar\delta(k_p)+\hbar g_rn_r +2\hbar g n_0\right)^2 - \left(\hbar g n_0\right)^2,
\end{equation}
From this we obtain the two branches of the so called bogoliubov dispersion relation in the free falling frame :

\begin{equation}
    \begin{aligned}
    \omega^{'\pm}_B(k)&=\pm\sqrt{\left(\dfrac{\hbar(k-k_p)^2}{2\mlp}-\delta(k_p)+ g_rn_r +2g n_0\right)^2 - \left(g n_0\right)^2} \\
    &=\pm\sqrt{\left(\dfrac{\hbar k'^2}{2\mlp}-\delta(k_p)+ g_rn_r +2g n_0\right)^2 - \left(g n_0\right)^2}.
    \end{aligned}
\end{equation}
while in the laboratory frame it yields :

\begin{equation}
    \begin{aligned}
    \omega^{\pm}_B(k)&=\vbf_0\cdot(\kbf-\kbf_p)+\omega^{'\pm}_B(k) \\
            &=\vbf_0\cdot(\kbf-\kbf_p)\pm\sqrt{\left(\dfrac{\hbar(k-k_p)^2}{2\mlp}-\delta(k_p)+ g_rn_r +2g n_0\right)^2 - \left(g n_0\right)^2}.
    \label{eq:bogo_lab_frame}
    \end{aligned}
\end{equation}
This expression is exactly what we would have obtained by directly applying a Galilean transformation to the spectrum in the free falling frame. With this form
the dual structure of the solution space exhibited earlier appear less clear. In fact, the motion of the fluid breaks the symmetry and the bogoliubov matrix no longer respects \autoref{eq:symetry_bog_2}. However, while changing the reference frame doppler shifts the eigenvalues, it leaves the eigenvectors untouched. More precisely, if $(u_{k'},v_{k'})^{\mathrm{T}}$ is an eigevector with eigenvalues $\hbar\omega_k'$ in the fluid frame it is also an eigenvector 
in the lab frame for the shifted frequency. As a consequence, the norm of the eigenvectors is the same and our description of the solution space remains relevant.

\subsubsection{Static fluid}


Let us start our description with the case of a static fluid. The dispersion is the same in both reference frame which can be seen by setting $\kbf_p=0$ 
in \autoref{eq:bogo_lab_frame} :

\begin{equation}
    \omega^\pm_B(k)= \pm\sqrt{\left(\dfrac{\hbar k^2}{2\mlp}-\delta(k_p)+ g_rn_r +2g n_0\right)^2 - \left(g n_0\right)^2}.
\end{equation}
This expression is in general drastically different from conservative systems where the Bogoliubov spectrum is linear. This feature is a direct consequence of 
interactions and thermodynamical equilibrium which imposes that the energetic cost of adding a particles to the system $\mu$ is equal
to the interaction energy due to the presence of the other bosons. Whenever the system is perturbed, it is thus more energetically favorable 
to dissipate energy through a collective excitation -- phonons -- rather than to add a particle to the condensate.

The polaritons dynamics is instead dictated by the interplay between losses, pumping and interactions. As a consequence, the 
collective excitations are not necessarily sound waves and can be massive or even energetically unstable. Indeed, 

\subsubsection{Bogoliubov modes energy}
The gravitationnal Hawking effect involve the emission of particles with opposite energy and momentum at the horizon. In our analog
system we demonstrated the existence of modes with opposite frequency and norm. However, the concept of "opposite" 

\subsubsection{Quantization of the fluctuations}
So far, we described the fluctuations of the order parameter as classical fields. This approach
show that the bogoliubov modes can exist in the system as solutions to the equation of motion. Nonetheless, this framework require 
an external perturbation for the modes to be excited