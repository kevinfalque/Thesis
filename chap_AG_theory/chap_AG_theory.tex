

% !TeX encoding = UTF-8
% !TeX spellcheck = fr_FR
% !TeX root = ../mythesis.tex
% !TeX program = pdflatex (build)
%%% TeXmaker : no 'magic comments' but set Root with Options > Set as master file

%useful stuff for what follows

\newcommand{\xbf}{\pmb{x}}
\newcommand{\ci}{\mathrm{i}}
\newcommand{\ee}{\mathrm{e}}
\newcommand{\lr}[1]{\left(#1\right)}
\newcommand{\lrsq}[1]{\left[#1\right]}
\newcommand{\tp}{\mathrm{p}}
\newcommand{\tv}{\mathrm{v}}
\newcommand{\vtv}{\boldsymbol{\mathrm{v}}}
\newcommand{\vna}{\boldsymbol{\nabla}}
\newcommand{\vx}{\mathbf{x}}
\newcommand{\tx}{\mathrm{x}}
\newcommand{\vobf}{\pmb{v_0}}
\newtheorem{theorem}{Theorem}
\newtheorem{lemma}[theorem]{Lemma}

\newcommand{\kbf}{\pmb{k}}
\newcommand{\vbf}{\pmb{v}}
\newcommand{\rbf}{\pmb{r}}
\newcommand{\ombog}{\omega_{\mathrm{b}}}

\newcommand{\rmexp}{\mathrm{exp}}
\newcommand{\im}{\mathfrak{Im}}
\newcommand{\re}{\mathfrak{Re}}
\newcommand{\mbogo}{m_{\mathrm{det}}}
\newcommand{\cs}{c_{\mathrm{s}}}
\newcommand{\dpsi}{\delta\psi}
\newcommand{\opsi}{\hat{\psi}}
\newcommand{\odpsi}{\hat{\delta\psi}}
% \newcommand{\norm}[1]{\left\lVert#1\right\rVert}

\newcommand{\Lbog}{\mathcal{L}_B}


\graphicspath{{./}{./fig/}{./chap_Ag_theory/fig/}}

\chapter{Hawking radiation in a polariton quantum fluid}
\label{chap:AG_theory}

Now that polaritons have been introduced, we can start to discuss the theoretical framework that will be used to 
describe Hawking radiation in a polariton quantum fluid. In this chapter, we will first establish the original hydrodynamical analogy between the propagation 
of acoustic waves in a moving fluid and scalar fields in curved spacetime. This analogy is at the heart of what motivated the study 
of Hawking radiation in analog systems, leading to a first variety of experiment in classical systems such as water tanks. These great experiments,
\cite{rousseaux_observation_2008,weinfurtner_measurement_2011} demonstrated positive to negative frequency conversion of shallow surface waves at the vicinity of a sonic horizon. However the temperature
of classical fluids is too high to quantize their collective excitation field which prevent the study of quantum effects. More precisely, Hawking radiation is expected 
to create entanglement between the emitted modes which can not be tested with water waves. Quantum fluids appear then as promising candidate since the 
fluctuations around the ground or steady state require a quantum treatment to be understood. In fact we will see that tackling a fluid with trans-critical flow with the usual Bogoliubov theory of condensed matter also predict particle creation from vacuum. Studying the effect from this point of view will reveal a strong robustness 
of the effect beyond the original hydrodynamical approach and widen the range of regimes where Hawking radiation can be observed.


\section{The hydrodynamical analogy}

In the previous chapter, we established that the dynamics of microcavity polaritons can be described by a driven dissipative Gross-Pitaevskii equation.
By writting the wavefunction in term of phase and density the Gross-Pitaevskii equation can be cast into a continuity equation and an Euler equation for a non-barotropic fluid in the pump rotating frame. Starting from the GPE
a system pumped by a laser of the general form $F_{\tp}e^{i\phi(x,t)}$ lead the fluid to a stationnary state whose wavefunction $\psi(x,t)=\sqrt{n_0}e^{i\phi(x,t)}$ phase and modulus respect :
\begin{equation}
    \begin{align}
    \partial_t \phi +& \frac{\mlp v_0^2}{2\hbar}+\frac{\hbar}{2\mlp}\frac{\partial_x^2\sqrt{n_0}}{\sqrt{n_0}}+V_{LP}+gn+g_rn_r+\frac{\re\lrsq{F_{\tp}e^{-i\theta}}}{\mlp}=0\,\\
    \partial_t n_0 +& \partial_x(n_0v)= \gamma n_0 -2\im\lrsq{F_{\tp}e^{-i\theta}}\sqrt{n_0}
    \end{align}
\end{equation}
If we neglect the pump and dissipation terms which is basically the situation of a conservative fluid, we recover the original equation describing 
sound waves in a convergent flow. This is precisely the equation that was used in the initial analogy made by W. Unruh \cite{unruh_experimental_1981}.
However, the out of equilibrium nature of the microcavity polaritons brings different phenomenology. In particular, perturbations propagating in
the fluid are not necessary sound waves as we will se latter. It means that the dispersion of these perturbation is not linear and 
can remarkably exhibit a gap which can be associated to massive excitations. To avoid making approximation and encapsulate as much as possible 
the complexity of the system it is convenient to make the same calculation in the pump rotating frame. It equivalently means that all frequencies
will be taken with respect to the pump. In practice this is done by writting the pump $F_{\tp}(x)= F_{\tp}e^{i\theta_p(x)}$ and look for steady state wavefunction 
of the same form $\psi(x,t)=\sqrt{n_0}e^{i\theta_p(x)}$. Plugging this into the GPE we obtain that the phase and density of the fluid must fullfill the following equations :

\begin{equation}
    \lrsq{-\frac{\hbar}{2\mlp}\nabla^2-i\frac{\hbar}{2}(\vna\cdot\pmb{v_0}) -i\hbar(\pmb{v_0}\cdot\vna)- \delta(v_0)+g_{\rm r}n_{\rm r}+g n_0-\ci\frac{\hbar\gamma}{2}}\sqrt{n_0}+|F_\tp|=0\,,
        \label{eq:StatHomogEOSDens}
\end{equation}
with $\hbar \delta(v_0)=\hbar\omega_\tp-\hbar\omega_0-\mlp v_0^2/2$ (note that this is just $\delta(k_p)$ in a homogeneous configuration where $\pmb{v_0}=\hbar\pmb{k_p}/\mlp$).
To study the low energy collective perturbations of the fluid state we linearize the system around the steady state by writing the wavefunction
$\psi(x,t)=(\sqrt{n_0}+\delta\psi(x,t))e^{i\theta_p(x)}$. Injecting this expression in \autoref{reservoir_eq} and using \autoref{eq:StatHomogEOSDens} we obtain at linear order in $\delta\psi$ :

\begin{equation}
    \ci \hbar \lr{\partial_t+\pmb{v_0}\cdot{\vna}}\delta\psi=\lrsq{-\frac{\hbar}{2m^*}\nabla^2+\rho-\ci\sigma}\phi+g n_0 \delta\psi^*,
        \label{eq:DefPertGPE}
\end{equation}
where $$\rho\coloneqq2g n_0- \delta(v_0)+g_{\rm r}n_{\rm r}\qquad\text{and}\qquad\sigma\coloneqq\hbar/2\vna \cdot\pmb{v_0}.$$
Note that here we have assumed that there is no external potential without loss of generality since it can be included in the definition of $\rho$.

\textcolor{red}{Ici il faut finir le calcul et notamment cette histoire de masse et de Klein gordon de mes couilles}

\section{At the heart of particle creation : Bogoliubov transformation}

\subsection{The ambiguity of vacuum definition}
The quantum vacuum state is defined as the quantum state of a system with the lowest possible energy. Eventhough 
this definition is commonly used and looks quite intuitive, it in fact implies some deep properties. First, a minimum energy state exists. In a finite dimension
problem, this is obvious since the spectrum has also a finite dimension but whenever the dimension of the Hilbert space is infinite, the existence of a minimum energy state is a priori not guaranteed.
Secondly, it seems to depend on the system which suggests that the vacuum is not a universal concept. This is particularly true in the context of quantum field theory in curved spacetime where the vacuum state is observer dependent.
For instance, in the Unruh effect, an accelerating observer detect a thermal radiation while an inertial observer sees 'standard' vacuum. This ambiguity is
as at the root of particle creation phenomena like the Hawking effect and once it is understood, make these effects appear less surprising.


\subsubsection{The harmonic oscillator}

To start with, let us consider the usual harmonic oscillator describing a massive particle in a one dimensionnal harmonic potential and
see how the defintion vacuum arise in this simple case \cite{CCT_tome1}. The Hamiltonian of the particle of mass $m$ is :


\begin{equation}
    H=\frac{P^2}{2m}+\frac{1}{2}m\omega^2X^2
\end{equation}
Where $P$ and $X$ are the momentum and position operators satisfying the canonical commutation relation $[X,P]=i\hbar$.
It is convenient to define the dimensionless operators :

\begin{subequations}
    \begin{align}
        \hat{P}&=\frac{1}{\sqrt{\hbar m\omega}}P,\\
        \hat{X}&=\sqrt{\frac{m\omega}{\hbar}}X, 
    \end{align}
\end{subequations}
which respect the commutation relation $[\hat{X},\hat{P}]=i$. The Hamiltonian is then written as :
\begin{equation}
    H=\hbar\omega\hat{H}
\end{equation}
with :
\begin{equation}
    \label{eq:dimensionless_hamiltonian}
    \hat{H}=\frac{1}{2}\lr{\hat{P}^2+\hat{X}^2}.
\end{equation}
The corresponding eigenproblem is usually tackled by introducing the creation and annihilation operators :

\begin{subequations}
    \begin{align}
        \hat{a}&=\frac{1}{\sqrt{2}}\lr{\hat{X}+i\hat{P}},\\
        \hat{a}^\dagger&=\frac{1}{\sqrt{2}}\lr{\hat{X}-i\hat{P}},
    \end{align}
\end{subequations}
which satisfy the commutation relation $[\hat{a},\hat{a}^\dagger]=1$.
Taking the definition of $\hat{a}$ and $\hat{a}^\dagger$ we look a the quantity $\hat{a}^\dagger\hat{a}$ :

\begin{equation}
    \begin{align}
    \hat{a}^\dagger\hat{a}&=\frac{1}{2}\lr{\hat{X}-i\hat{P}}\lr{\hat{X}+i\hat{P}} \\
                          &=\frac{1}{2}\lr{\hat{X}^2+\hat{P}^2+i\lr{\hat{X}\hat{P}-\hat{P}\hat{X}}} \\
                          &=\frac{1}{2}\lr{\hat{X}^2+\hat{P}^2-1}.
    \end{align}
\end{equation}
Comparing with \autoref{eq:dimensionless_hamiltonian} we see that :

\begin{equation}
    \label{eq:ham_number_operator}
    \hat{H}=\hat{a}^\dagger\hat{a}+\frac{1}{2}.
\end{equation}
We then naturally introduce the operator $\hat{N}=\hat{a}^\dagger\hat{a}$. Eventhough this operator 
is known to be the number operator, we will on purpose treat it without prior knowledge and see what can be learned from the derivation.
From \autoref{eq:ham_number_operator} we see that the an eigenstate $\ket{\phi^i_\nu}$ of $\hat{N}$ with eigenvalue $\nu$ is also
an eigenstate of $\hat{H}$ with eigenvalue $\nu+\frac{1}{2}$. Solving the problem then boils down to find the spectrum of $\hat{N}$.

\textbf{Spectrum determination.} The operator $\hat{N}$ is hermitian and its eigenstates form a complete basis of the Hilbert space. Let us 
show two usefull lemmas resulting directly from the definition of $\hat{N}$ :

\begin{lemma}
    The eigenvalues of $\hat{N}$ are positive or zero.
\end{lemma}

\begin{proof}
    Let us consider any eigenstate $\ket{\phi_\nu}$ of $\hat{N}$. The square of norm of the state $\hat{a}\ket{\phi_\nu}$ is 
    \begin{equation}
            \norm{\hat{a}\ket{\phi_\nu}}^2=\bra{\phi_\nu}\hat{a}^\dagger\hat{a}\ket{\phi^i_\nu} \geq 0   
\end{equation}
    Using the definition of $\hat{N}$ we have :
    \begin{equation}
        \label{eq:positive_eigenvalue}
        \bra{\phi_\nu}\hat{a}^\dagger\hat{a}\ket{\phi_\nu}=\nu\bra{\phi_\nu}\ket{\phi_\nu} \geq 0.
    \end{equation}
Since $\bra{\phi_\nu}\ket{\phi_\nu}>0$ we have $\nu\geq 0$.
\end{proof}

\begin{lemma}
    Let $\ket{\phi_\nu}$ be an eigenstate of $\hat{N}$ with eigenvalue $\nu$.
    \begin{itemize}
        \item \ $\nu=0$ if and only if $\hat{a}\ket{\phi_\nu}=0$.
        \item If $\nu>0$ then $\hat{a}\ket{\phi_\nu}$ is an eigenstate of $\hat{N}$ with eigenvalue $\nu-1$.
    \end{itemize}
\end{lemma}

\begin{proof}
    If $\nu=0$, \autoref{eq:positive_eigenvalue} implies that $\bra{\phi_\nu}\ket{\phi_\nu}=0$ which imposes that $\ket{\phi_\nu}=0$.
    Reciprocally, if $\hat{a}\ket{\phi_\nu}=0$ a multiplication by $\hat{a}^\dagger$ gives $\hat{N}\ket{\phi^i_\nu}=0$. Since $\ket{\phi_\nu}$ is not zero we conclude
    that $\ket{\phi_\nu}$ is an eigenvector of $\hat{N}$ with eigenvalue $\nu =0$.

    COnsider now that $\nu>0$, we now know that $\hat{a}\ket{\phi_\nu}$ is not zero. By using the identity $\bigl[\hat{N},\hat{a}\bigr] = -\hat{a}$ we 
    obtain :
    \begin{equation}
        \hat{N}\hat{a}\ket{\phi_\nu}=\hat{a}\hat{N}\ket{\phi_\nu}-\hat{a}\ket{\phi_\nu}=(\nu-1)\hat{a}\ket{\phi_\nu},
    \end{equation}
    which shows that $\hat{a}\ket{\phi_\nu}$ is an eigenstate of $\hat{N}$ with eigenvalue $\nu-1$.
\end{proof}

\noindent Let us summarize what we know so far :
\begin{itemize}
    \item The operator $\hat{N}$ has a non negative spectrum. 
    \item If $\nu$ is an eigenvalue of $\hat{N}$ then $\nu-1$ is also an eigenvalue of $\hat{N}$.
\end{itemize}

This two properties are in fact sufficient to show that the spectrum of $\hat{N}$ are the positive or zero integers. \\
\bigskip 

Assume, by contradiction, that there is a positive non integer eigenvalue $\nu$ associated with the eigenvector $\ket{\phi_\nu}$ . Then, by the second lemma, $\nu-1$ is also a non integer eigenvalue of $\hat{N}$ generated by 
the action of $\hat{a}$ on $\ket{\phi_\nu}$. By iterating this process we obtain a sequence of eigenvalues $\nu,\ \nu-1,\ \nu-2,\dots$ which is impossible since the spectrum is bounded from below. More precisely, it exists
a positive integer $p$ such that $\nu-p<0$ which contradicts the fact that the spectrum is non negative. We conclude that the spectrum of $\hat{N}$ is the positive or zero integers.




\section{Elementary excitations of the fluid : Bogoliubov theory}

In contrast with classical systems where Hawking radiation can only originate from disturbance external to the system, the analogous Hawking 
radiation in quantum fluids may be seeded by vacuum fluctuations. Perturbations of the quantum fluid are emitted at the horizon and propagate in opposite directions as it would 
happen in the vicinity of a gravitationnal event horizon. However, it is worth noticing that at this point, the analogy with black holes physics originating from 
hydrodynamical equations may be dropped and rather be seen as an inspiration to look for particle creation in a quantum fluid. Indeed, treating the collective excitations
of a transsonic Bose Einstein condensate with the Bogoliubov theory of condensed matter already predicts emission of mode at the horizon from vacuum. In fact, 
both the Stephen Hawking original calculation and the quantum fluid analog rely on a Bogoliubov transformation between 
two set of operators, \cite{hawking_black_1972} which ultimately leads to the mixing of creation and annihilation operators. This section will establish 
the strucutre of the collective excitation spectrum and show how a certain fluid configuration can lead to mode emission from vacuum. Furthermore,
we will show that the Hawking effect is robust against departures from the original hydrodynamical analogy and don't even require the excitations to be sound waves. The out 
of equilibrium nature of the polariton system will indeed reveal a much richer phenomenology that conservative systems extending the possibility to observe 
Hawking radiation of massive excitations.

\subsection{Homogeneous fluid}

Let us consider a monochromatic pump described by a plane wave $F_{\tp}(\rbf,t)=F_{\tp}e^{i(\kbf_p \rbf -\omega_pt)}$. As explained in the previous chapter, this will drive the system
to a steady state of the form $\psi_0(\rbf,t)=\psi_0e^{i(\kbf_p \rbf -\omega_pt)}$ and respecting :

\begin{equation}
    \left[\omp -\omlp - \dfrac{\hbar \kp^2}{2 \mlp} - g \abs{\psilp^0}^2 + i \dfrac{\gamlp}{2} \right] \psilp^0= \eta_{LP} F_p^0.
    \label{eq:steady_state}
\end{equation}

We now look for the elementary excitations following the Bogoliubov prescription which consists in linearizing the GPE around $\psi_0$. We write the wavefunction as :

\begin{equation}
    \psi(\rbf,t)=\biggl[\psi_0(\rbf)+\dpsi(\rbf,t)\biggr]e^{i(\kbf_p \rbf -\omega_pt)}.
    \label{eq:linearization_ansatz}
\end{equation}

The second term in this writting can be interpreted as the polaritons that are not in the steady state and therefore have a global phase different from the pump. 
This imply that they can have different energy and momentum that the fluid and undergo scatterings or spontaneous effects since their dynamics is not fixed by the pump. It is 
worth noticing that this term do not originate from the out of equilibrium nature of the system and would be not zero even at zero temperature due to interactions.
In the case of Bose Einstein condensates, this linearization process actually introduces a state that is not physical in the sense that it suggest the breaking of the $U(1)$ symmetry of the system or equivalently 
 that the condensates picked a phase \cite{castin_bose-einstein_2001}. The true physical state is recovered through a statistical mixtures of all these "symetry broken states" which then rather appear 
as intermediate state convenient for calculation. In the case of a polariton fluid pumped quasi resonantly the situation is different since the mean field phase 
is fixed by the pump. 

\bigskip

Injecting \autoref{eq:linearization_ansatz} in the GPE, using \autoref{eq:steady_state} and keeping only the linear terms in $\dpsi$ we obtain :

\begin{equation}
    \begin{split}
        i  \hbar \partial_t \dpsi =& \biggl(\hbar\omlp^0 - \hbar \omp+\dfrac{\hbar^2}{2\mlp}\left[\nabla^2 +2i\kbf_p\nabla-\kbf_p^2\right]+\hbar g_rn_r + 2\hbar g \abs{\psi_0}^2+ \dfrac{i\hbar\gamma}{2} \biggr) \dpsi \\
                                    &+ \hbar g \psi_0^2 \dpsi^*. 
    \end{split}
    \label{eq:bogo_v1}
\end{equation}

Yet, this equation is not stricly speaking linear in $\dpsi$ since there is a term involving its complex conjugated $\dpsi^*$. A way to solve this problem
is to decompose $\dpsi$ in its real and imaginary part and to obtain two independant equations. An equivalent procedure, more common in the litterature, is to find an equation on 
the complex conjugated and consider $\dpsi^*$ as an independant variable. This is the approach we will follow here. The equation on $\dpsi^*$ is obtained by simple complex conjugation of \autoref{eq:bogo_v1} :

\begin{equation}
    \begin{split}
    -i  \hbar \partial_t \dpsi^* = &\biggl(\hbar\omlp^0 - \hbar \omp+\dfrac{\hbar^2}{2\mlp}\left[\nabla^2 -2i\kbf_p\nabla-\kbf_p^2\right]+\hbar g_rn_r + 2\hbar g \abs{\psi_0}^2- \dfrac{i\hbar\gamma}{2}\biggr) \dpsi^* \\
      &+ \hbar g \psi_0^{*2} \dpsi.
    \end{split}
    \label{eq:bogo_v2}
\end{equation}
For the sake of clarity we define the operator :
\begin{equation}
    \Ham_{bog}= \hbar\omlp^0 - \hbar \omp+\dfrac{\hbar^2}{2\mlp}\left[\nabla^2 +2i\kbf_p\nabla-\kbf_p^2\right]+\hbar g_rn_r + 2\hbar g \abs{\psi_0}^2+\dfrac{i\hbar}{2}.
    \label{eq:hambog}
\end{equation}
We can now wirte the fully linearized problem under its matrix form :

\begin{equation}
    \begin{matrix}i\hbar \partial_t
        \begin{pmatrix}
             \dpsi \\
             \dpsi^*
        \end{pmatrix}
        = \Lbog
        \begin{pmatrix}
             \dpsi \\
             \dpsi^*
        \end{pmatrix},
    \end{matrix}
    \label{eq:bogo_matrix}
\end{equation}
where $\Lbog$ is the Bogoliubov matrix defined as :
\begin{equation}
    \begin{matrix}
    \Lbog =
    \begin{pmatrix}
        \Ham_{bog} &  \hbar g \psi_0^2\\
        -\hbar g \psi_0^{*2} & -\Ham_{bog}^*
    \end{pmatrix}
\end{matrix}
\end{equation}

\bigskip 

\noindent \textbf{Bogoliubov matrix symetries.} As in general for quadratic hamiltonian of bosonic systems \cite{castin_bose-einstein_2001} the Bogoliubov matrix is not hermitian. Therefore its eigenvalues are not 
necessarily real and $\Lbog$ is not even ensured to be diagonalizable. However, it can easily be shown that the following symetry is respected :

\begin{equation}
    \Lbog^\dagger = \eta^{-1} \Lbog \eta \ \ \ \mathrm{with} \ \ \ \eta = \eta^{-1} =\begin{pmatrix}
        1 & 0 \\
        0 & -1
    \end{pmatrix}.
    \label{eq:symetry_bog}
\end{equation}

It means that the Bogoliubov operator is "hermitian" for the inner product :

\begin{equation}
    \langle \vec{X_1},\vec{X_2} \rangle = \vec{X_1}^\dagger \cdot \eta \vec{X_2},
    \label{eq:inner_product}
\end{equation}
where $cdot$ is the usual euclidian scalar product. It means that for any vectors $\vec{X_1}$ and $\vec{X_2}$ we have $\langle \vec{X_1},\Lbog\vec{X_2} \rangle = \langle \Lbog\vec{X_1},\vec{X_2} \rangle^*$. 
