%!TeX encoding = UTF-8
% Metadata for coverpages
\reference{} %To be completed after defense
%%SU  %%
\atinstitution{Sorbonne Université}
\atinstitutionacro{SU}
\univ{de Sorbonne Université}
%%  PSL  %%
%\univ{Université PSL}
%\atinstitution{à l'École normale supérieure}
%\atinstitutionacro{ENS}
%%
\atlab{au Laboratoire Kastler Brossel}
\logolab{logo_lkb}                      
\thesisname{Thèse de doctorat}
\gradename{docteur}
\specialite{Physique}
\ecoledoctnum{564}
\ecoledoct{Physique en Île-de-France}
\author{Kevin FALQUE}
\advisor{Alberto BRAMATI}
\titlefr{Etude expérimentale de la radiation de Hawking stimulée et correlations entre modes de Bogoliubov dans un fluide quantique de polaritons.}
\titleen{Experimental study of stimulated Hawking radiation and Bogoliubov modes correlations in a quantum fluid of polaritons.}
% si nécessaire, pour les métadonnées
\titlemetafr{Etude expérimentale de la radiation de Hawking stimulée et correlations entre modes de Bogoliubov dans un fluide quantique de polaritons.} 
\titlemetaen{Experimental study of stimulated Hawking radiation and Bogoliubov modes correlations in a quantum fluid of polaritons.}
\dateiso{2025-07-04} % YYYY-MM-DD
\defensecity{à Paris}



% Jury like LaTeX Tabular with \& as separators;  no president defined before defence 
\jury{
M. & Maxime & Richard & DR & CNRS & Rapporteur    \\
M. & Alesssandro & Fabbri & PR & Univ. de Valencia & Rapporteur \\
\Mme& Carole & Diederichs & PR & ENS Paris & Examinatrice \\
M. & Chris & Westbrook & DR & Institut d'optique & Examinateur \\
M. & Nicolas & Pavloff & PR & Univ. de Paris Saclay & Examinateur \\
M. & Alberto & Bramati & PR & SU & Directeur de thèse 
}

\resume{Ce manuscrit est dédié à l'étude expérimentale de la radiation de Hawking stimulée dans un fluide quantique de polaritons en microcavité ainsi qu'au correlations d'intensité entre modes d'excitations collectives du système.

Dans un premier temps, nous établissons le cadre théorique nécessaire à la compréhension des fluides de polaritons. En partant de l'équation de Gross-Pitaevskii, nous dérivons le spectre de Bogoliubov des excitations collectives et analysons les régimes dynamiques en fonction de la densité, de la vitesse et des interactions du fluide. Cette analyse met en évidence le rôle clé des modes d'énergie positive et négative dans la création de particules analogues à l'effet Hawking.

Ensuite, nous présentons la réalisation expérimentale d'un écoulement transcritique dans un fluide de polaritons, obtenue grâce à un façonnage précis du pompage optique. Cette configuration permet de créer une région transcritique, essentielle pour l'apparition d'un horizon sonore. Nous démontrons également la présence de modes d'énergie négative dans cette région, confirmant les conditions nécessaires à l'observation de l'effet Hawking. Cette étude permet notamment d'étendre le cadre dans lequel
il est possible d'observer des phenomènes de creation de particules à des configurations où le spectre de Bogoliubov n'est pas necessairement linéaire et comprend des excitations massives.

Sur cette base, nous réalisons l'observation expérimentale de l'émission stimulée du rayonnement de Hawking. En injectant un état cohérent dans la région amont et en mesurant les modes diffusés, nous mettons en évidence une signature du mélange des énergies positives et négatives à l'interface. Bien que nos résultats ne constituent pas encore une mesure quantitative complète de la matrice de diffusion, ils démontrent la puissance des techniques optiques utilisées pour reconstruire le champ de fluctuations et analyser les modes diffusés.
    
Enfin, nous explorons les corrélations entre les modes d'excitations collectives dans le cas d'un fluide au repos en vue d'une étude plus approfondie des aspects quantiques de l'effet Hawking. Nous discutons également des perspectives offertes par les fluides de polaritons pour des expériences de gravité analogue, grâce à leur nature optique, leur caractère hors équilibre et leur grande flexibilité.

Cette thèse présente la première mesure expérimentale de l'effet Hawking stimulé et met en évidence le potentiel des systèmes polaritoniques comme plateforme expérimentale pour étudier des phénomènes fondamentaux de la physique des champs quantiques en espace temps courbés, ouvrant ainsi la voie à de nouvelles investigations sur les effets quantiques aux horizons analogiques.
}

\motscles{Optique quantique, fluides quantiques de lumière, superfluidité,
contrôle tout optique, gravité analogue, rayonnement de Hawking, création de particules, théorie des champs, espace-temps courbe, trous noirs, corrélation, detection balancée}

\abstract{This thesis is devoted to the experimental investigation of stimulated Hawking radiation in a quantum fluid of polaritons within a microcavity, as well as the analysis of intensity correlations between collective excitation modes in the system.
Initially, a comprehensive theoretical framework is established to describe polariton fluids. Beginning with the Gross-Pitaevskii equation, the Bogoliubov spectrum of collective excitations is derived, enabling a detailed examination of the dynamic regimes as a function of fluid density, velocity, and interaction strength. This theoretical analysis underscores the important role of positive and negative energy modes in the formation of particle analogs associated with the Hawking effect.


Subsequently, the thesis presents the experimental realization of a transcritical flow in a polariton fluid, achieved through precise optical pumping engineering. 
This configuration facilitates the formation of a transcritical region, a necessary condition for the emergence of a sonic horizon. 
The presence of negative energy modes in this region is experimentally confirmed, verifying the conditions required for the manifestation of the Hawking effect. Notably, this study extends the scope of particle creation phenomena to configurations where the Bogoliubov spectrum is not strictly linear and yields massive excitations.

Building upon these findings, the thesis reports the experimental observation of stimulated Hawking radiation. By injecting a coherent state into the upstream region and analyzing the scattered modes, a signature of energy mixing at the interface between positive and negative modes is identified. 
While these results do not yet constitute a comprehensive quantitative measurement of the scattering matrix, they demonstrate the efficacy of optical techniques in reconstructing the fluctuation field and characterizing the scattered modes.

Lastly, the thesis explores the correlation dynamics between collective excitation modes in a stationary fluid as a preliminary step toward a more rigorous investigation of the quantum aspects of the Hawking effect. 
Additionally, it discusses the potential of polariton fluids as a versatile experimental platform for analog gravity studies, given their optical nature, non-equilibrium dynamics, and high tunability.

Overall, this work presents the first experimental demonstration of stimulated Hawking radiation in a polariton system and underscores the viability of such systems for probing fundamental phenomena in quantum field theory within curved spacetime, thus opening new avenues for the exploration of quantum effects at analog horizons.

}

\keywords{Quantum optics, quantum fluids of light, quantum fluids, superfluidity, 
full optical control, analogue gravity, hawking radiation , particles creation, quantum field theory, curved spacetime, black holes, correlation, balanced detection}

%% to reduce font size in abstract/resume add at the begining 
%% \fontsize{<fontsize>}{<baselineskip>}\selectfont  
%% where <fontsize> and <baselineskip> are integers (eg 10 and 11)