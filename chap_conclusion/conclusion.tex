% !TeX encoding = UTF-8
% !TeX spellcheck = fr_FR
% !TeX root = ../mythesis.tex
% !TeX program = pdflatex (build)
%%% TeXmaker : no 'magic comments' but set Root with Options > Set as master file

%useful stuff for what follows

\chapter{Conclusion}

In this thesis, we have explored the physics of microcavity polaritons and their application to analog gravity, with a particular focus on the Hawking effect in polariton quantum fluids. By combining theoretical and experimental approaches, we have demonstrated the versatility of this platform for studying curved spacetime and analog gravity effects.

We began by presenting the microcavity polariton system, detailing its unique properties as a hybrid light-matter quasiparticle. This was followed by a theoretical investigation of the Hawking effect in polariton quantum fluids, where we showed how the Bogoliubov excitations in these systems can mimic the behavior of quantum fields in curved spacetime. This theoretical framework provided the foundation for designing experiments to study analog gravity effects.

A key achievement of this work was the demonstration of the tunability of polariton systems, which allowed us to effectively create curved spacetime geometries. By carefully controlling the system parameters, we were able to engineer conditions that emulate event horizons and study the associated analog gravity phenomena. This tunability highlights the potential of polariton systems as a flexible platform for exploring a wide range of analog gravity effects.

We then reported the first observation of stimulated Hawking radiation in a polariton quantum fluid. By injecting a weak probe beam, we stimulated the emission of Bogoliubov modes and observed their properties, providing direct evidence of the analog Hawking effect in this system. This result represents a significant step forward in the experimental study of analog gravity.

Finally, we focused on the measurement of correlations between Bogoliubov modes. By analyzing the intensity correlations, we gained insights into the underlying quantum processes driving the analog Hawking effect. This work paves the way toward the measurement of quantum correlations between modes emitted by an analog horizon, a crucial step for exploring the quantum nature of analog gravity phenomena.

In conclusion, this thesis has demonstrated the potential of microcavity polariton systems as a platform for studying analog gravity and quantum field theory in curved spacetime. The results presented here open new avenues for investigating quantum correlations and non-classical effects in analog horizons, bridging the gap between condensed matter physics and fundamental questions in quantum gravity.